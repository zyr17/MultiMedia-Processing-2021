% \iffalse meta-comment
%
% Copyright (C) 2017--2020 by Xiangdong Zeng <xdzeng96@gmail.com>
%
% This work may be distributed and/or modified under the
% conditions of the LaTeX Project Public License, either
% version 1.3c of this license or (at your option) any later
% version. The latest version of this license is in:
%
%   http://www.latex-project.org/lppl.txt
%
% and version 1.3 or later is part of all distributions of
% LaTeX version 2005/12/01 or later.
%
% This work has the LPPL maintenance status `maintained'.
%
% The Current Maintainer of this work is Xiangdong Zeng.
%
% \fi

%*********************************************************************
% fduthesis: 复旦大学论文模板
% 2020/08/30 v0.7e + 2021/12/20 期末论文魔改版
%
% 重要提示:
%   1. 请确保使用 UTF-8 编码保存
%   2. 请使用 XeLaTeX 或 LuaLaTeX 编译
%   3. 请仔细阅读用户文档
%   4. 修改、使用、发布本文档请务必遵循 LaTeX Project Public License
%   5. 不需要的注释可以尽情删除
%*********************************************************************

%\documentclass{fduthesis}
% 模板选项:
%   type = doctor|master|bachelor  论文类型,默认为本科论文
%   oneside|twoside                论文的单双面模式,默认为 twoside
%   draft = true|false             是否开启草稿模式,默认关闭
% 带选项的用法示例:
%   \documentclass[oneside]{fduthesis}
%   \documentclass[twoside, draft=true]{fduthesis}
\documentclass[type=master, twoside, draft=false]{fduthesis}

\fdusetup{
  % 参数设置
  % 允许采用两种方式设置选项:
  %   1. style/... = ...
  %   2. style = { ... = ... }
  % 注意事项:
  %   1. 不要出现空行
  %   2. “=” 两侧的空格会被忽略
  %   3. “/” 两侧的空格不会被忽略
  %   4. 请使用英文逗号 “,” 分隔选项
  %
  % style 类用于设置论文格式
  style = {
    font = times,
    % 西文字体(包括数学字体)
    % 允许选项:
    %   font = garamond|libertinus|lm|palatino|times|times*|none
    %
    %cjk-font = fandol,
    cjk-font = windows,
    % 中文字体
    % 允许选项:
    %   cjk-font = adobe|fandol|founder|mac|sinotype|sourcehan|windows|none
    %
    % 注意:
    %   1. 中文字体设置高度依赖于系统。各系统建议方案:
    %        windows:cjk-font = windows
    %        mac:    cjk-font = mac
    %        linux:  cjk-font = fandol(默认值)
    %   2. 除 fandol 和 sourcehan 外,其余字体均为商用字体,请注意版权问题
    %   3. 但 fandol 字体缺字比较严重,而 sourcehan 没有配备楷体和仿宋体
    %   4. 这里中西文字体设置均注释掉了,即使用默认设置:
    %        font     = times
    %        cjk-font = fandol
    %   5. 使用 font = none / cjk-font = none 关闭默认字体设置,需手动进行配置
    %
    font-size = -4,
    % 字号
    % 允许选项:
    %   font-size = -4|5
    %
    % fullwidth-stop = catcode,
    % 是否把全角实心句点 “.” 作为默认的句号形状
    % 允许选项:
    %   fullwidth-stop = catcode|mapping|false
    % 说明:
    %   catcode   显式的 “。” 会被替换为 “.”(e.g. 不包括用宏定义保存的 “。”)
    %   mapping   所有的 “。” 会被替换为 “.”(使用 LuaLaTeX 编译则无效)
    %   false     不进行替换
    %
    footnote-style = xits,
    % 脚注编号样式
    % 允许选项:
    %   footnote-style = plain|libertinus|libertinus*|libertinus-sans|
    %                    pifont|pifont*|pifont-sans|pifont-sans*|
    %                    xits|xits-sans|xits-sans*
    %
    % hyperlink = color,
    % 超链接样式
    % 允许选项:
    %   hyperlink = border|color|none
    %
    % hyperlink-color = default,
    % 超链接颜色
    % 允许选项:
    %   hyperlink-color = default|classic|elegant|fantasy|material|
    %                     business|science|summer|autumn|graylevel|prl
    % 默认与西文字体保持一致
    %
    bib-backend = bibtex,
    % 参考文献支持方式
    % 允许选项:
    %   bib-backend = bibtex|biblatex
    %
    % bib-style = numerical,
    % 参考文献样式
    % 允许选项:
    %   bib-style = author-year|numerical|<其他样式>
    % 说明:
    %   author-year  著者—出版年制
    %   numerical    顺序编码制
    %   <其他样式>   使用其他 .bst(bibtex)或 .bbx(biblatex)格式文件
    %
    % cite-style = {},
    % 引用样式
    % 默认为空,即与参考文献样式保持一致
    % 仅适用于 biblatex;如要填写,需保证相应的 .cbx 格式文件能被调用
    %
    bib-resource = {cite,papers},
    % 参考文献数据源
    % 可以是单个文件,也可以是用英文逗号 “,” 隔开的一组文件
    % 如果使用 biblatex,则必须明确给出 .bib 后缀名
    %
    % logo = {fudan-name.pdf},
    % 封面中的校名图片
    % 模版已自带,通常不需要额外配置
    %
    % logo-size = {0.5\textwidth},      % 只设置宽度
    % logo-size = {{}, 3cm},            % 只设置高度
    % logo-size = {8cm, 3cm},           % 设置宽度和高度
    % 设置校名图片的大小
    % 通常不需要调整
    %
    % auto-make-cover = true
    % 是否自动生成论文封面(封一)、指导小组成员名单(封二)和声明页(封三)
    % 除非特殊需要(e.g. 不要封面),否则不建议设为 false
  },
  %
  % info 类用于录入论文信息
  info = {
    title = {大规模多模态预训练算法调研},
    % 中文标题
    % 长标题建议使用 “\\” 命令手动换行(不是指在源文件里输入回车符,当然
    % 源文件里适当的换行可以有助于代码清晰):
    %   title = {最高人民法院、最高人民检察院关于适用\\
    %            犯罪嫌疑人、被告人逃匿、死亡案件违法所得\\
    %            没收程序若干问题的规定},
    %
    title* = {Survey on Large-Scale Multi-Modal Pre-training Methods},
    % 英文标题
    %
    author = {蒋骐泽},
    % 作者姓名
    %
    % author* = {Your name},
    % 作者姓名(英文 / 拼音)
    % 目前不需要填写
    %
    major = {多媒体信息处理与分析},
    % 课程名称
    %
    student-id = {21110270042},
    % 作者学号
    %
    date = {2021 年 12 月 29 日},
    % 日期
    % 注释掉表示使用编译日期
    %
  }
}

% 需要的宏包可以自行调用
%\usepackage{physics}
\usepackage{comment}
\usepackage{color}
\usepackage[ruled,noend]{algorithm2e}
\SetAlgorithmName{算法}{algorithmautorefname}{}
\usepackage{multirow}
\usepackage[figuresright]{rotating}
\usepackage{listings}
\usepackage{lscape}
\usepackage{graphicx}

% 需要的命令可以自行定义
%\newcommand{\hilbertH}{\symcal{H}}
%\newcommand{\ee}{\symrm{e}}
%\newcommand{\ii}{\symrm{i}}
\newcommand{\qize}[1]{{\color{red} {\textbf{#1}}}}

\newtheorem{problem}{问题}
\newtheorem{proposition}{命题}
\newtheorem{assumption}{假设}

\begin{document}

% 这个命令用来关闭版心底部强制对齐,可以减少不必要的 underfull \vbox 提示,但会影响排版效果
% \raggedbottom

% 前置部分包含目录、中英文摘要以及符号表等
\frontmatter

% 目录
\tableofcontents
% 插图目录
%\listoffigures
% 表格目录
% \listoftables


\begin{abstract}
  
  深度神经网络相关算法在近期得到了长足的发展。在计算机视觉方向,深度神经网络算法在目标分类、目标识别等领域取得了突破性进展。在自然语言处理方向,深度神经网络也被大量用于机器翻译、文本分类、语义理解等任务。
  
  近年来,随着博客、短视频、自媒体等多媒体多模态的用户生成内容(User Generated Content, UGC)急剧增加,如何联合分析和利用不同模态的大规模数据成为了一个新的热门研究方向。相比单模态任务,多模态任务需要将不同模态的数据统一学习到一个表示空间,从而能够跨模态进行语意理解、跨模态召回等任务。
  
  本文对计算机视觉和自然语言处理研究方向上的重要进展进行了总结,针对调研了视觉模态和文本模态大规模融合学习的相关工作,并从多个角度总结了目前的研究工作和进展。最后,我们探讨了该问题的未来研究方向。
  
  本文的主要内容包括:
  
  \begin{enumerate}
      \item 介绍在计算机视觉和自然语言处理方向的重要进展。
      
      \item 给出多模态问题的定义,并给出该问题的常见学习目标。
      
      \item 从多个方向总结目前学术界在多模态任务上的最新进展。
      
      \item 分析目前多模态任务上面临的挑战和目前工作的不足,展望该问题的未来研究方向。
      
  \end{enumerate}
  

\end{abstract}


% 符号表
% 语法与 LaTeX 表格一致:列用 & 区分,行用 \\ 区分
% 如需修改格式,可以使用可选参数:
%   \begin{notation}[ll]
%     $x$ & 坐标 \\
%     $p$ & 动量
%   \end{notation}
% 可选参数与 LaTeX 标准表格的列格式说明语法一致
% 这里的 “ll” 表示两列均为自动宽度,并且左对齐
%\begin{notation}[ll]
%  $x$                  & 坐标        \\
%  $p$                  & 动量        \\
%  $\psi(x)$            & 波函数      \\
%  $\bra{x}$            & 左矢(bra) \\
%  $\ket{x}$            & 右矢(ket) \\
%  $\ip{\alpha}{\beta}$ & 内积        \\
%\end{notation}

% 主体部分是论文的核心
\mainmatter

% 建议采用多文件编译的方式
% 比较好的做法是把每一章放进一个单独的 tex 文件里,并在这里用 \include 导入,例如
%   \include{chapter1}
%   \include{chapter2}
%   \include{chapter3}


\chapter{引言}\label{chap:introduction}

\section{研究背景和意义}\label{sec:background}

随着互联网浪潮的推进,在近几年互联网用户急速攀升,
据中国互联网络信息中心\footnote{http://www.cnnic.net.cn/,访问时间2021年12月24日}统计,
2018年至2021年间,中国网民数量从57.7\%提升到71.6\%,规模达到10.07亿,
同时其中99.6\%的网民使用手机上网。快速增长的网民群体,
尤其是手机网民群体促进了移动互联网的快速发展,这同时也使用户生成内容(User Generated Content)种类和数量的大幅上升。
从最早的博客、微博,到后来的公众号、UP主,再到现在占据主流的短视频,UGC的内容丰富度越来越高,
创作门槛也在逐步降低。曾经需要内容生产者到台式电脑前剪辑、编辑才能创作,
变为了任何人使用手机就能快速制作不同风格的作品。

UGC内容的快速增加,对内容分发平台提出了挑战。平台需要充分利用不同模态的大规模UGC内容,
发掘内容之间的联系。
以目前流行的短视频平台为例,一个短视频包含视频本身、标题、标签、评论等图文信息,
是一种典型的多模态数据源。在一个短视频的生命周期中,有大量的任务需要依赖对短视频的图文理解和模态关联。
例如在视频创作阶段,视频制作平台需要根据用户填写的短视频标题和拍摄内容,向用户推荐热门标签和图文素材。
同时,平台也需要对视频内容进行安全、合规审查。
在视频上线后,平台需要将视频推荐给合适的用户,在搜索时能够准确查找相关视频,发现热点视频,
以及利用视频信息提高相关广告和电商推送质量。
这一切都离不开对短视频多模态数据的分析和融合。只有同时利用多个模态的数据,
才能更好的发掘视频的完整信息,同时支持跨模态的查询、理解,提升平台质量。

在计算机视觉和自然语言处理方向,深度神经网络已经成为了主流的方法。
然而,如何融合学习视觉模态和文本模态的内容并未被充分的研究。
随着UGC内容的快速增加,该问题的重要性也得到了学术界和工业界的广泛关注。

多模态融合学习具有以下重要优势:

\begin{itemize}
    \item 通过融合多个模态的数据,模型得到的表示相较单模态具有更丰富的特征层次。
    \item 融合了多模态数据,从而可以完成跨模态的下游任务。
    \item 在数据存在模态缺失时,模型能够通过其他模态产生信息互补。
\end{itemize}

多模态融合学习也具有以下关键难点:

\begin{itemize}
    \item 如何处理来自不同模态的异构输入,并得到统一表示。
    \item 如何使语义相近数据的向量表示在特征空间中也相近。
    \item 在模态缺失的情况下,如何利用已有模态数据得到较完整的特征表达,以及补全缺失模态。
\end{itemize}

本文对今年图文模态融合的工作开展了调研,总结了目前的研究现状,并探讨了该问题的未来研究方向。

本文的论文结构如下:

\begin{itemize}
    \item 第\ref{chap:related}章介绍了计算机视觉和自然语言处理方向近年来的重点工作。
    多模态算法大多基于这两个方向上的先进工作,并着重研究模态间的融合方式,
    因此是进行多模态问题研究的重要基础。
    \item 第\ref{chap:definition}章给出了多模态融合学习的问题定义,
    介绍了常见模态数据来源和下游任务场景。
    \item 第\ref{chap:survey}章从模型结构、目标问题、使用方法、数据源四个方面归纳总结目前的多模态融合工作。
    \item 第\ref{chap:conclusion}章对多模态问题及其方法进行总结,并探讨了未来工作方向。
\end{itemize}

\begin{comment}

\section{研究内容与技术路线}

\subsection{信号灯控制问题概述}

\end{comment}


\chapter{计算机视觉和自然语言处理问题的研究进展}\label{chap:related}

在多模态融合问题中,最常使用、数据最丰富的就是视觉模态和文字模态。
在这两个方向上,深度神经网络算法成为了近年来的主流,
同时取得了一系列的突破性进展。在大规模多模态数据融合问题中,
离不开这两个研究方向上的成果。
本章会分别总结视觉模态和文字模态两个方向上深度学习模型的重要成果,
同时简单介绍使用大规模数据进行预训练的概念和优势。

\section{计算机视觉深度学习模型}\label{sec:cvmodel}

计算机视觉是一个经典的机器学习问题。随着AlexNet在ImageNet任务上取得了突破性成绩,
深度神经网络成为了解决计算机视觉问题的主流方法。

AlexNet\cite{krizhevsky2012imagenet}是最早在ImageNet上使用深度神经网络取得显著成功的算法,
并在之后掀起了深度神经网络热潮。AlexNet由5个卷积神经网络层(CNN)和三个全连接网络层(FC)组成,
同时使用ReLU作为激活函数,避免了梯度消失和梯度爆炸问题;使用dropout方法减轻网络的过拟合。
该方法在ImageNet 2010比赛中取得了62.5\%的分类准确率,显著高于其他方法。

VGG网络\cite{simonyan2015deep}相比于AlexNet增加了很多卷积层,
但同时减少了卷积核的大小。该网络证明了当卷积神经网络层数够多时,
较小的卷积核也能学习到大范围的图像特征。

ResNet\cite{he2015deep}对卷积神经网络层使用残差结构,
将卷积神经网络得到的结果累加到经过卷积层之前的结果上,
有效缓解了深度神经网络层数太多出现的梯度消失和梯度爆炸问题,
大大增加了神经网络深度,在保持网络参数量较少的同时提升模型效果。

Vision Transformer(ViT)\cite{dosovitskiy2020image}是一种基于Transformer的视觉深度学习模型。
和之前基于CNN的模型不同,模型使用了基于注意力机制的Transformer结构,
使模型能够感知到完整的图像,比起CNN方法具有更好的整体性,
但同时这也导致模型参数量较大、容易过拟合。因此ViT方法虽然在数据量较小时表现不够出色,
但是在数据量足够的情况下能够取得比基于CNN的网络结构取得更好的效果。

\section{自然语言处理深度学习模型}\label{sec:cvmodel}

自然语言处理任务也是最早尝试使用深度学习模型解决的任务之一。
在自然语言处理任务中,模型获取一句话作为输入,并完成语义理解、
情感分析、文本生成、机器翻译等任务。

由于语言中词语是离散的,因此如何将离散的词语连续表示,
即词语的向量化是自然语言处理任务中的关键问题。
Word2Vec\cite{mikolov2013efficient}是该方向上的一个经典工作。
该方法可以使用大规模语料进行训练,提升了模型效果。
在学习目标上,该方法通过使相邻词语的向量表示更加相近来完成,
并提出了CBOW和skip-gram两种方法。

由于自然语言天然带有顺序关系,因此循环神经网络成为了处理自然语言问题的一种直观方法。
其中长短时记忆神经网络(LSTM)\cite{hochreiter1997long}是一种常用的改进型循环神经网络。
LSTM可以更好的应对梯度消失和梯度爆炸问题,使用双向LSTM还可以同时处理出现在左侧和右侧的内容。

Seq2Seq模型\cite{sutskever2014sequence}则在机器翻译领域取得了显著成功。
该模型首先使用作为编码器的神经网络对数据文本进行处理得到文本表示,
然后再使用一个解码器网络根据编码器学到的文本表示预测输出文本,
并将预测结果作为下一次的输入。

近年来在自然语言处理领域最振奋人心的成就是大规模预训练模型BERT\cite{devlin2019bert}的提出。
BERT是一个使用多层自注意力机制(self-attention)\cite{vaswani2017attention}的神经网络,
该模型使用了亿级别的无标注wikipedia文本数据作为预训练数据,并设计了两个训练任务:
Masked-Language Modeling (MLM)和Next Sentence Prediction (NSP),
使模型能够根据句子的上下文对遮盖词语进行预测,以及推测两个句子的前后关系。
在完成预训练后,BERT可以接入各种不同的下游任务,对下游任务简单微调就可以取得十分优秀的效果。



\subsection{预训练学习模型}

随着计算力、训练数据规模和模型效果三个方面的稳步提升,出现了一种新的学习方式:预训练学习。
预训练学习利用海量无标注的数据,学习数据间的特点,将数据进行向量表示。
广泛的实验表明,在数据量大幅提升的情况下,由于数据分布更接近于真实分布,
同时模型过拟合的可能性也显著降低,因此可以显著增强模型对数据的提取能力。
使用预训练学习方法,可以使主干网络(backbone network)有更好的表示能力,
从而提升模型在下游任务上的表现。

\chapter{多模态融合学习的问题定义}\label{chap:definition}

在本章,我们给出多模态融合学习问题的定义,介绍多模态任务的数据来源和其使用多模态融合学习的下游任务场景。

\section{多模态融合学习的定义}\label{sec:definition}

多模态融合学习由``多模态''、``融合''和``学习''三个词构成,因此该问题有以下三个特点:

\begin{enumerate}
    \item 在该任务中,存在\textbf{多种不同模态的输入}。常见的模态例如图像模态、
    文本模态、音频模态、视频模态、用户画像模态等。
    \item 该任务需要\textbf{融合来源于不同模态的数据}。
    虽然存在多种不同模态,模型需要将多种不同模态统一表示,
    使模态之间可以互相关联。
    \item 模型利用多模态数据,\textbf{学习模态间的关联和统一表示},
    从而可以使用学习到的多模态表示完成多种下游任务。
\end{enumerate}

\section{多模态融合学习任务的数据来源}

多模态融合学习任务的数据必定有至少两个模态。多模态学习任务常见的数据来源如下:

\begin{itemize}
    \item \textbf{图像-标签数据}是最常见的一种多模态数据来源。
    深度神经网络大放异彩的ImageNet任务就是一种典型的图像-标签数据。
    在该任务中,图像属于视觉模态,标签一般会被作为文本模态。
    由于文本模态大多为完整的句子,所以将标签作为文本模态时,
    常会将标签转换为句子(例如标签``Bird''会转换成``There is a bird in the picture.''),
    尤其是训练或者测试包含多种来源不同的数据集的时候。
    \item \textbf{图像-文本数据}是最常用的多模态数据之一。
    相比图像-标签数据,该数据使用一句或多句句子来描述图片内容。
    该数据和图像-标签数据相比的明显优势在于不一定需要人工标注,
    从而可以通过互联网大规模收集,获取海量数据。例如通过Google等搜索引擎或是Tumblr等社交网站的搜索和热点功能,
    可以快速收集到大量具有相关性的图像-文本数据对。
    \item \textbf{视频-文本数据}是近年来快速增长的多模态数据类型。
    随着视频网站,尤其是短视频的兴起,用户创造了大量视频。
    视频包含视频标题和视频内容,同样构成了多模态数据。
    \item \textbf{视频-音频数据}也属于一种多模态数据。
    在使用这种类型的数据时,常常会使用语音识别技术将音频转换为文本模态再加以利用。
    但是音频也包含音乐、情感等独特的信息,也是一种独特的模态。
    \item \textbf{更多模态混合的数据。}除了上述提到的几种主流的多模态数据,
    还有更多包含多种模态的数据。例如一个视频包含视频、标签、标题文本、语音、
    发布者用户画像等大量模态可以利用,一篇博客也会包含图像、文本、标题、标签、
    评论、作者画像等模态。
\end{itemize}

在各种模态中,由于视觉模态(图像,视频)-文本模态(文字,标签,语音识别)的多模态数据是最常见、
丰富的,目前的多模态融合学习方法研究的也以这两个模态为主。

\section{多模态融合学习任务的下游任务场景}

多模态融合学习是十分重要的研究方向,具有许多重要下游应用场景。
本节给出了多模态融合学习最常见的一些下游应用任务。

\begin{itemize}
    \item \textbf{多模态召回任务。}该任务是一个典型的跨模态任务,
    任务目标是给出多模态数据的其中一个模态,要求模型从大量另一模态的数据中召回和自己对应的数据。
    该任务最常见于跨模态搜索、跨模态推荐等实际场景。
    \item \textbf{图像标注任务。}该任务是一种跨模态生成任务。
    给定视觉模态数据,模型需要生成一段文字来描述该数据。
    \item \textbf{图像识别任务。}该任务是非常常见的任务之一,模型需要识别图像中的物体名称。
    图像分类也和该任务类似,需要识别某个图像属于哪一类。
    \item \textbf{视觉问题回答(Visual Question Answering)。}该任务会同时输入两个模态的数据——
    图像和问题,并要求模型回答正确的答案。该任务在智能对话机器人中应用广泛。
    \item \textbf{多语言任务。}在文本模态,可能包含不同国家语言的文本,它们也类似于多个模态的数据。
    在多语言任务中常见的有机器翻译、知识图谱迁移等场景。
\end{itemize}


\chapter{多模态融合工作调研}\label{chap:survey}

在本章,我们总结调研了目前的多模态融合学习工作。
在该问题中,Transformer是最流行的结构,几乎所有工作都使用了该结构。
Transformer结构在多模态融合任务上有下述优势:

\begin{itemize}
    \item Transformer模型学习能力强,在大规模训练数据时表现尤为突出。
    由于多模态融合数据量都很大,因此该结构效果很好。
    \item Transformer结构在视觉和自然语言处理方向都取得了成功,
    使用该结构可以更好的利用已有成果。
\end{itemize}

同时,随着数据规模的不断增大,以及Transformer结构在预训练任务上取得的成功,
目前该方向的工作几乎都采用了大规模数据预训练,并在之后微调或是零样本学习的方式。
预训练方法是指通过使用大规模数据进行训练,不针对某个特定下游任务,
使模型能够学习到对训练数据的一个较好的表示,从而经过简单的线性探测,甚至是零样本测试,
在下游任务上取得更好的效果。

% Please add the following required packages to your document preamble:
% \usepackage{multirow}
% \usepackage{graphicx}
\begin{sidewaystable}[]
\resizebox{\textwidth}{!}{%
\begin{tabular}{l|c|l|l|l|l}
\hline
工作 & \multicolumn{1}{l|}{融合方法} & 预训练任务 & 模态编码网络 & 下游任务 & 预训练数据集 \\ \hline
VideoBERT\cite{videobert} & \multirow{11}{*}{单流} & MLM, SIP & S3D & $z$-CLS, VC & YouCook II \\
VIsualBERT\cite{visualbert} &  & MLM, SIP & RCNN, ResNet & VQA, VCR, NLVR, REC & COCO \\
VL-BERT\cite{vlbert} &  & MLM, MOC & RCNN, ResNet & VQA, VCR, REC & CC, BooksCorpus, $eng$-Wikipedia \\
Oscar\cite{oscar} &  & MLM, Contrastive & RCNN, ResNet & VQA, NLVR, I2TR, IC & COCO, CC, SBU, M30K, GQA \\
UNITER\cite{uniter} &  & MLM, SIP, WRA, MRM, MRFR, MOC & RCNN & VQA, VCR, NLVR, I2TR, T2IR, REC & COCO, VG, CC, SBU \\
Unicoder-VL\cite{unicodervl} &  & MLM, SIP, MOC & RCNN & I2TR, $z$-I2TR, VCR & CC, SBU \\
M3P\cite{m3p} &  & MLM, SIP, MRM, MMLM & RCNN & $ml$-I2TR & CC, $ml$-Wikipedia \\
UC$^2$\cite{uc2} &  & MLM, SIP, MRM, EA, VTLM & RCNN & $ml$-I2TR, $ml$-VQA & M30K, COCO \\
M6\cite{m6} &  & MLM, MMLM, ISG & / & VQA, IC, I2TR, TG, T2IG & 非公开 \\
ViLT\cite{vilt} &  & MLM, SIP & / & VQA, NLVR, I2TR, T2IR & COCO, VG, SBU, GCC \\
SOHO\cite{soho} &  & MLM, SIP & ResNet & VQA, NLVR, I2TR & COCO, VG \\ \hline
ViLBERT\cite{vilbert} & \multirow{10}{*}{双流} & MLM, SIP, MOC & RCNN, ResNet, BERT & VQA, VCR, REC, T2IR, $z$-T2IR & CC \\
LXMERT\cite{lxmert} &  & MLM, SIP, MOC, MRFR & RCNN, ResNet, BERT & VQA, NLVR & COCO \\
CBT\cite{cbt} &  & Contrastive & S3D, BERT & VC, AR, AS & Kinetics \\
CLIP\cite{clip} &  & Contrastive & ResNet/ViT, BERT & $z$-CLS, CLS & 非公开 \\
12-in-1\cite{12in1} &  & MLM, SIP, MOC & RCNN, ResNet, BERT & VQA, NLVR, REC, T2IR & CC \\
WenLan\cite{wenlan} &  & Contrastive & RCNN, RoBERTa & IC, I2TR, T2IR & 非公开 \\
ALIGN\cite{align} &  & Contrastive & EfficientNet, BERT & I2TR, T2IR, I2IR, T2TR, CLS & 非公开 \\
COOKIE\cite{cookie} &  & Contrastive & ResNet, BERT & I2TR, I2TR, T2TR, CLS & CC, SBU, COCO, F30K, VQA 2.0, GQA \\
ClipBERT\cite{clipbert} &  & Contrastive & ResNet, BERT & T2IR VQA & COCO, VG \\
ZeroVL\cite{zerovl} &  & Contrastive & ViT/SwinT, BERT & T2IR, I2TR, $z$-T2IR, $z$-I2TR & CC, CC12M, SBU, VG, 非公开 \\ \hline
\end{tabular}%
}
\caption{近期基于Transformer的视觉-文本模态融合工作}
\label{tab:overall}
\end{sidewaystable}

表\ref{tab:overall}中,我们总结了近期使用Transformer结构的视觉-文本模态融合工作,
以及它们的重要特征。针对这些工作的异同,我们在
第\ref{sec:singledouble}节讨论了该任务上两种不同的多模态融合思路,
第\ref{sec:pretrain}节介绍了训练时不同方法使用的预训练任务,
第\ref{sec:backbone}节根据模型使用的编码网络进行分类,
第\ref{sec:downstream}节说明了模型应用的下游任务,
第\ref{sec:dataset}节统计了不同工作使用的数据来源。

\section{单流和双流多模态融合学习}\label{sec:singledouble}

在多模态融合学习中,主要分为了单流模型和双流模型两种方法。
两种方法最大的区别是:单流模型中,两个模态从一开始就进行融合,
最后得到一个统一表示。双流模型中,两个模态则首先分别经过两个独立的编码模块,
再将编码后的结果互相交互,最终得到融合表示。
从工作的分布来看,在多模态融合任务中单流和双流均有大量工作。

单流的工作中,VideoBERT\cite{videobert}是该方向的开山之作之一,
建立了单流多模态融合问题的基本思路。该方向文章的主要思路为:

\begin{itemize}
    \item 获取多模态数据,通过模态编码网络或不进行处理,输入融合Transformer。
    \item 设计预训练任务,使模型能够学习到模态间信息,并得到多模态数据的表示。
    \item 使用线性探测或零样本学习,将学习到的表示用于下游任务。
\end{itemize}

然而,单流模式也有其难点。首先,视觉模态和文本模态输入之间差距巨大,
在设计预训练任务时却大多按照BERT的训练模式,忽略了模态间的关联和差异性,
导致信息泄露,或者是提高模型学习难度。其次,单流模型中的Transformer大多使用在文本上预训练的参数初始化,
但是这和图片输入完全不匹配,大量新增的图片输入可能会让模型学习困难,
甚至对文本表示也产生影响。在接下来的工作中,也大多朝着这两个方向进行优化,
设计了更贴合多模态的预训练任务,以及对文本和图片模态做更显著的区分来改善模型学习效果。

双流多模态融合学习也是一个主流方向。由于双流模型首先使用各自模态的编码网络将模态输入转换为向量表示,
该模型可以较为简单的更换编码模型,从而更容易使用单模态研究取得的新进展。
同时,使用双流多模态融合学习时,一般会对模态编码模型进行微调,
由于模型可以利用多个模态数据,这一般也会使编码模型生成的编码质量有所提升。

ViLBERT\cite{vilbert}是该方向的重要工作,它首先得到不同模态数据的向量表示,
然后使用互注意力方法,得到质量更好的模态表示向量。
在对比学习提出后,由于该学习方式在超大规模数据上具有出色的效果,
在双流学习方法上也被广泛采用。
该方向文章的主要思路为:

\begin{itemize}
    \item 获取多模态数据,通过模态编码网络得到模态向量表示。
    \item 学习模态表示间的交互方法,从而将多模态数据表示融合,或是表示于同一空间。
    \item 使用线性探测或零样本学习,将学习到的表示用于下游任务。
\end{itemize}

双流模式也存在一些困难。由于模态是分别经过各自的编码网络的,
虽然这带来了替换编码网络的方便性,但是也使网络结构较为分散,
可能会影响最终效果。

\section{预训练任务}\label{sec:pretrain}

本节会介绍不同工作涉及的预训练任务。部分工作的预训练任务可能会有微小不同,
为了统一在本质不变的情况下归为一类。
\begin{itemize}
    \item \textbf{Masked Language Modeling (MLM)。}该任务是BERT使用的预训练任务之一,
    所有使用Transformer融合结构的方法均使用了该任务。
    该任务会在训练时随机将部分词语替换成\verb|[MASK]|,或是随机的一个词,
    并要求模型能够在经过Transformer后将词语还原。该方法也包含众多变种,
    例如遮挡部分必须连续\cite{m6}、只对文本进行遮挡\cite{visualbert}、图片遮挡时作用于全输入\cite{vlbert},
    遮挡文本时同时遮挡相关标签\cite{oscar}等。
    在多语言学习任务中,
    还存在两个类似的变种任务,分别为\textbf{Visual Translation Language Modeling (VTLM)}和\textbf{Multimodality Masked Language Modeling (MMLM)}。
    前者使用图像模态作为输入,需要同时还原两个不同语言文本;后者则使用图像和其他语言文本作为输入还原文本。
    
    \item \textbf{Sentence-Image Prediction (SIP)。}该方法类似于BERT中的下一句预测(NSP),
    给定两个模态输入,判断这两个模态是不是来自于同一个数据。
    
    \item \textbf{Masked Object Classification (MOC) / Masked Region Modelling (MRM) / Masked Region Feature Regression (MRFR)。}
    这些目标概念基本相同,在包含目标检测的方法中,
    许多方法采用了该任务,对被遮盖物体进行分类或者对被遮盖的向量回归。其本质上和MLM相同,
    都是遮盖了数据的部分信息并让模型尝试还原。
    
    \item \textbf{Word-Region Alignment (WRA)。}该任务需要将文本中的单词和对应的目标检测框对齐。
    这有助于模型能够理解图像和文本之间的对应关系。
    
    \item \textbf{Image Sentence Generation (ISG)。}该任务是一个生成任务,根据图片生成图片的标题,
    与真实标题比较相似性。
    
    \item \textbf{Contrastive。}对比学习是近年来热门的方法。对比学习需要模型能够从海量负例中找到正例。
    在单模态对比学习中,一般会采用数据增强的方式构造正例。而在多模态任务中,
    由于不同模态的数据天然表示了同一概念,因此非常适合使用对比学习。
    使用这种方法的多为双流模式,使用该任务约束两个模态的向量表示,让语义越接近的模态数据其向量表示也越接近。
    对向量距离进行度量时,常用简单的向量夹角距离。该方法目标上类似于Triplet Loss,
    但是在效果上和Triplet Loss相比占有明显优势。在UC$^2$中使用了\textbf{Early Adaption (EA)}方法,
    和对比学习类似,它使用KL散度约束图像和文本的向量位于同样空间。
\end{itemize}

\section{模态编码网络}\label{sec:backbone}

由于涉及到多个模态的输入,多模态融合模型一般会使用每个模态对应的编码网络,
从而得到一个模态更好的表示,便于进一步学习。其中单流工作一般直接将文本模态直接输入Transformer模型,
而对视觉模态使用编码网络。在双流工作中,则会对视觉和文本两个模态都使用编码网络。

在视觉编码网络中,主要涉及S3D、RCNN、ResNet和ViT这四类网络。\textbf{S3D}\cite{xie2018rethinking}是一种针对视频的编码网络,
可以编码视频的时序信息,被部分视频-文本多模态学习的方法使用。但是由于视频模态的数据量相比图片模态具有明显劣势,
导致视频模态的预训练编码网络质量较图像模态不占优势,在ClipBERT\cite{clipbert}等方法中,
虽然也研究了视频-文本多模态学习,却仍使用图像模态编码网络(ResNet等),
并取得了更好的效果。\textbf{RCNN}\cite{ren2015faster}用于从图像中识别目标物体。
由于图像模态和文本模态的信息密度差距很大,因此很多方法,尤其是单流方法会先使用RCNN识别图像中的物体,
再将识别到的物体图像作为输入。这虽然降低了网络从图像中学习的难度,
但是因为物体识别网络不够准确,也可能影响最终效果。
在数据规模较大的时候,大多不会使用目标识别网络辅助。
\textbf{ResNet}\cite{he2015deep}是基于卷积神经网络的,在视觉模态最流行的网络之一,
绝大部分方法都使用了该网络进行图像模态的编码,以及用于识别目标物体。
它也具有很多变种和改进型,在表\ref{tab:overall}中的EfficientNet就是其中之一。
\textbf{ViT}\cite{dosovitskiy2020image}是今年来成功将Transformer结构用于图像模态任务的成功代表,
相比基于卷积神经网络的模型,在训练数据量较大的时候会更具优势。
Swin Transformer (SiwnT)\cite{liu2021swin}是由微软提出的该方法的一种改进版本。

在双流模式中,方法还需要文本编码网络。几乎所有方法都使用了BERT\cite{devlin2019bert}作为其编码网络,
这说明了在文本编码领域其领先地位。也有部分方法使用了BERT的改进型RoBERTa\cite{liu2019roberta}作为编码网络。

\section{下游任务}\label{sec:downstream}

% Please add the following required packages to your document preamble:
% \usepackage{graphicx}
% \usepackage{lscape}
\begin{table}[]
\begin{tabular}{c|c|c}
\hline
简称 & 全称 & 中文名 \\ \hline
I2TR & Image-to-Text Retrieval & 图像到文本召回 \\
T2IR & Text-to-Image Retrieval & 文本到图像召回 \\
I2IR & Image-to-Image Retrieval & 图像到图像召回 \\
T2TR & Text-to-Text Retrieval & 文本到文本召回 \\
CLS & Classification & 分类 \\
AR & Action Recognition & 视频动作识别 \\
AS & Action Segmentation & 视频动作分段 \\
VQA & Visual Question Answering & 视觉问题回答 \\
REC & Referring Expression Comprehension & 指代目标理解 \\
VCR & Visual Commonsence Reasoning & 视觉常识推理 \\
NLVR & \multicolumn{1}{l|}{Natural Language for Visual Reasoning} & \multicolumn{1}{l}{用于视觉推理的自然语言} \\
IC & Image Captioning & 图像标题生成 \\
VC & Video Captioning & 视频标题生成 \\
T2IG & Text-to-Image Generation & 文本到图像生成 \\
TG & Text Generation & 文本生成 \\
\hline
\end{tabular}
\caption{下游任务简称和对应名称}
\label{tab:downstream}
\end{table}

为了检验预训练学习结果,需要将模型应用在下游任务上测试性能。
常见的测试方法是使用所学模型生成多模态数据的向量表示,然后训练线性探测模块得到任务结果。
也有部分模型尝试了零样本(zero-shot)方法,在不经过额外训练的情况下直接测试模型效果,
以证明模型学习到表示的泛用性。在下游任务中,如果带有$z$-前缀,
表示该任务是零样本测试的,如果带有$ml$-前缀,说明任务在本身的基础上还涉及多语言场景,
例如$ml$-I2TR任务会使用图像模态输入在多种不同语言中召回对应文本。
表\ref{tab:overall}中下游任务简称的意义见表\ref{tab:downstream}。
下游任务主要可以分为如下几类:

\begin{itemize}
    \item \textbf{召回任务(I2TR, T2IR, I2IR, T2TR)。}召回任务是多模态学习中最传统的任务之一。
    模型得到一个模态的锚点数据,并需要从另一个模态的所有候选数据中找到和锚点最接近的数据。
    在单流模型中类似SIP预训练任务,在双流模型中类似Contrastive任务。
    
    \item \textbf{分类任务(CLS, AR, AS)。}该任务是单模态中的传统任务,
    以ImageNet数据集为例,模型需要获取图片,并给出该图片所属的类别。
    由于多模态任务是使用图像-文本的训练方式,在完成分类任务时,
    一般会将图像对应类别使用模板学习(prompt learning)方法转换为句子,
    例如类别bird会转换成There is a bird in the picture。
    对于每个类别构造对应的句子,并使用召回任务的思路就可以得到分类结果。
    
    \item \textbf{问题回答(VQA, REC, VCR, NLVR)。}该任务会给出视觉模态的数据和文本模态的问题,
    模型需要根据视觉信息和问题给出问题的回答。这个任务检测了模型的多模态语言理解能力。
    
    \item \textbf{生成任务(IC, VC, T2IG, TG)。}该任务获取一个模态的输入,
    并生成未输入模态的结果。
\end{itemize}

\section{预训练数据集}\label{sec:dataset}

在预训练中,数据集大小会非常显著的影响模型性能,尤其是零样本任务中\cite{align}。
表\ref{tab:dataset}给出了不同数据集的数据量大小。
大部分方法都说明了使用的数据集规模。
有部分方法除了公开数据集外,还使用了部分网络收集的或是私有数据集,
我们也给出了这些方法相关数据集的规模。

\begin{table}[]
\centering
\begin{tabular}{c|cc}
数据集名称 & 图片/视频数量(个) & 文本数量(句) \\ \hline
YouCook II & 2K$^*$ & 10K \\
COCO & 113K & 565K \\
CC & 3M$^*$ & 3M \\
CC12M & 12M$^*$ & 12M \\
BooksCorpus & / & 74M \\
$eng$-Wikipedia & / & ? \\
$ml$-Wikipedia & / & 101G\footnote{111} \\
SBU & 1M$^*$ & 1M \\
F30K & 31K & 153K \\
M30K & 30K & 150K \\
VG & 108K & 5.4M \\
GQA & 113K & 22.7M \\
VQA 2.0 & 265K & 1.4M \\
GCC & 3M & 3M \\
Kinetics & 500K & 500K \\
M6-非公开数据 & 60M & 400M \\
CLIP-非公开数据 & 400M & 400M \\
WenLan-非公开数据 & 30M & 30M \\
ALIGN-非公开数据 & 1.8B & 1.8B \\
ZeroVL-非公开数据 & 100M & 100M
\end{tabular}
\caption{预训练数据集规模。标注了$^*$的数据集仅包含文本,需要自行下载图片,因此实际可用图片数量小于数据集创建时的大小。}
\label{tab:dataset}
\end{table}

可以看到,公开数据集最大规模的CC12M也仅有千万级图文,而非公开数据集则显著大于公开数据集。
尤其是ALIGN,使用了十亿级网络获取的图文对用于训练,也得到了极好的效果。
因此,在多模态预训练任务中,虽然提升模型效果是很重要的,
但是如果迫切需要提升模型性能,在计算资源和数据量都足够的情况下,
使用更多数据训练是最简便有效的方法,也因此最新效果提升明显的多模态融合工作多为企业发表。

% \include{tex/exp}

\chapter{总结与展望}\label{chap:conclusion}

通过对目前多模态融合领域文章的调研,我们可以发现:

\begin{itemize}
    \item 目前多模态融合研究主要涉及视觉模态和文本模态的双模态。
    \item 目前主流的工作都使用了Transformer为主要架构,采用预训练的学习方式,
    在模型上主要分为单流模型和双流模型两种。
    \item 多模态融合预训练在许多单模态和多模态任务上都取得了成功,
    具有较好的使用前景。
    \item 目前的研究表明,训练使用的数据量会对多模态融合模型的效果产生显著影响,
    提升数据量可以得到很好的效果。
\end{itemize}

基于目前的研究,在多模态融合领域还有很多可以开展的工作。
例如目前的多模态任务几乎都集中于视觉模态和文本模态,
如果结合利用其它模态数据是一个亟待研究的方向。
同时,目前的研究都只涉及两个模态的融合,在模态数更多时应该如何融合也很值得研究。
同时,在模型优化上,虽然通过增大训练数据量和模型大小可以取得明显的提升,
但是在部分场景可能不能很好的提升数据量的大小或是使用很大的模型,
需要考虑在保持数据规模和模态编码网络大小不增大的情况下如何提升预训练效果。

\begin{comment}
\begin{figure}
    \centering
    \includegraphics[width=46mm]{images/placeholder.png}
    %\vspace{-0.1in}
    \caption{123}
\end{figure}


\begin{figure}
    \centering
    \includegraphics[width=46mm]{images/placeholder.png}
    \includegraphics[width=23mm]{images/placeholder.png}
    %\vspace{-0.1in}
    \caption{456}
\end{figure}
\end{comment}

% 后置部分包含参考文献
\backmatter

% 打印参考文献列表
\printbibliography

\end{document}
